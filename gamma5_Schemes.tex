\documentclass[12pt]{article}

\usepackage[margin=1in]{geometry}
\setlength{\parindent}{1cm}
\setlength{\parskip}{.5em}
\renewcommand{\baselinestretch}{1.0}
\usepackage{tikz,multicol,amsfonts,relsize,amsmath,physics,graphicx,slashed,color,colortbl,wrapfig,mathrsfs,caption,subcaption,pgfplots,array,longtable,multirow,bm,stackengine,pgfplots,fancyhdr,titlesec,enumitem,hyperref,float,lastpage,mathtools,tensor,amssymb}
\pgfplotsset{compat=newest}
\hypersetup{
colorlinks=true,setpagesize,
linkcolor=black,
filecolor=magenta,      
urlcolor=cyan,
}
\definecolor{Gray}{gray}{0.3}
\usetikzlibrary{arrows,positioning} 
\usetikzlibrary{arrows.meta}
\numberwithin{equation}{section}
\numberwithin{figure}{section}
\numberwithin{table}{section}
\newcommand{\numberthis}{\addtocounter{equation}{1}\tag{\theequation}}
\newcommand{\numberthisa}{\addtocounter{equation}{1}\tag{\theequation a}}
\newcommand{\numberthisb}{\addtocounter{equation}{0}\tag{\theequation b}}
\newcommand{\numberthisc}{\addtocounter{equation}{0}\tag{\theequation c}}
\newcommand{\numberthisd}{\addtocounter{equation}{0}\tag{\theequation d}}
\newcommand{\numberthise}{\addtocounter{equation}{0}\tag{\theequation e}}
\newcommand{\numberthisf}{\addtocounter{equation}{0}\tag{\theequation f}}
\def\shrinkage{2.1mu}
\def\vecsign{\mathchar"017E}
\def\dvecsign{\smash{\stackon[-2.60pt]{\mkern-\shrinkage\vecsign}{\rotatebox{180}{$\mkern-\shrinkage\vecsign$}}}}
\def\dvec#1{\def\useanchorwidth{T}\stackon[-4.2pt]{#1}{\,\dvecsign}}
\stackMath
\tikzset{
	>=stealth',}
\renewcommand{\theequation}{\arabic{section}.\arabic{equation}}
\renewcommand{\thefigure}{\arabic{section}.\arabic{figure}}
\renewcommand{\thetable}{\arabic{section}.\arabic{table}}
\newcolumntype{C}[1]{>{\centering\arraybackslash}m{#1}}


\titleformat{\section}
{\Large\scshape}{\thesection}{1em}{}
\titleformat{\subsection}
{\large\scshape}{\thesubsection}{1em}{}
\titleformat{\subsubsection}
{\scshape}{\thesubsubsection}{1em}{}
\pagestyle{fancy}
\lhead{\(\gamma_5\) Schemes}
\rhead{\thepage/\pageref{LastPage}}
\cfoot{}
\renewcommand{\thesection}{\Roman{section}} 
\renewcommand{\thesubsection}{\thesection.\textup{\roman{subsection}}}
\renewcommand{\thesubsubsection}{\thesubsection.\textup{\roman{subsubsection}}}

\makeatletter
\renewcommand\@dotsep{1000}
\makeatother
\makeatletter
\renewcommand{\l@section}{\@dottedtocline{1}{1.5em}{2.6em}}
\renewcommand{\l@subsection}{\@dottedtocline{2}{4.0em}{3.0em}}
\renewcommand{\l@subsubsection}{\@dottedtocline{3}{7.4em}{4.5em}}
\makeatother

\newcommand{\ve}[2]{\left[
	\begin{array}{c}
		#1\\
		
		#2
	\end{array}
	\right]
}

\allowdisplaybreaks

\newcommand{\ndot}{\mathrel{\raisebox{-2.3pt}{$ \!  \mathlarger{\mathlarger{\mathlarger{\mathlarger{\cdot}}}} \! $}}}

\newcommand{\edot}{\mathrel{\raisebox{-2.3pt}{$\mathlarger{\mathlarger{\mathlarger{\cdot}}}$}}}

\newcommand{\Li}[1]{\mathrm{Li}_{#1}}

\newcommand{\twomatrix}[4]{
	\left[
	\begin{array}{cc}
		#1 & #2\\
		#3 & #4
	\end{array}
	\right]
}
\newcommand{\threematrix}[9]{
	\left[
	\begin{array}{ccc}
		#1 & #2 & #3\\
		#4 & #5 & #6\\
		#7 & #8 & #9
	\end{array}
	\right]
}
%-----------------------------------------------------------------------------------------
%tikz stuff
%-----------------------------------------------------------------------------------------

\newcommand{\tikzline}[4]{\draw (#1,#2) -- (#3,#4);}
\newcommand{\tikztline}[4]{\draw[thick] (#1,#2) -- (#3,#4);}
\newcommand{\tikzutline}[4]{\draw[ultra thick] (#1,#2) -- (#3,#4);}
\newcommand{\tikzaline}[4]{\draw[->] (#1,#2) -- (#3,#4);}
\newcommand{\tikzatline}[4]{\draw[thick,->] (#1,#2) -- (#3,#4);}
\newcommand{\tikzautline}[4]{\draw[ultra thick,->] (#1,#2) -- (#3,#4);}

%
% Polar input: takes in a starting point, a radial distance from that point, and an angle with the x-axis
%

\newcommand{\tikzlinep}[4]{\draw (#1,#2) -- (#1+#3*cos{#4},#2+#3*sin{#4});}
\newcommand{\tikztlinep}[4]{\draw[thick] (#1,#2) -- (#1+#3*cos{#4},#2+#3*sin{#4});}
\newcommand{\tikzutlinep}[4]{\draw[ultra thick] (#1,#2) -- (#1+#3*cos{#4},#2+#3*sin{#4});}
\newcommand{\tikzalinep}[4]{\draw[->] (#1,#2) -- (#1+#3*cos{#4},#2+#3*sin{#4});}
\newcommand{\tikzatlinep}[4]{\draw[thick,->] (#1,#2) -- (#1+#3*cos{#4},#2+#3*sin{#4});}
\newcommand{\tikzautlinep}[4]{\draw[ultra thick,->] (#1,#2) -- (#1+#3*cos{#4},#2+#3*sin{#4});}

\newcommand{\tikzsquare}[3]{\draw (#1,#2) -- (#1+#3,#2) -- (#1+#3,#2+#3) -- (#1,#2+#3) -- cycle;}
\newcommand{\tikztsquare}[3]{\draw[thick] (#1,#2) -- (#1+#3,#2) -- (#1+#3,#2+#3) -- (#1,#2+#3) -- cycle;}
\newcommand{\tikzutsquare}[3]{\draw[ultra thick] (#1,#2) -- (#1+#3,#2) -- (#1+#3,#2+#3) -- (#1,#2+#3) -- cycle;}

\newcommand{\tikzrect}[4]{\draw (#1,#2) -- (#1+#3,#2) -- (#1+#3,#2+#4) -- (#1,#2+#4) -- cycle;}
\newcommand{\tikztrect}[4]{\draw[thick] (#1,#2) -- (#1+#3,#2) -- (#1+#3,#2+#4) -- (#1,#2+#4) -- cycle;}
\newcommand{\tikzutrect}[4]{\draw[ultra thick] (#1,#2) -- (#1+#3,#2) -- (#1+#3,#2+#4) -- (#1,#2+#4) -- cycle;}

\newcommand{\tikzcircle}[3]{\draw (#1,#2) circle (#3);}
\newcommand{\tikztcircle}[3]{\draw[thick] (#1,#2) circle (#3);}
\newcommand{\tikzutcircle}[3]{\draw[ultra thick] (#1,#2) circle (#3);}

\newcommand{\tikzellipse}[4]{\draw (#1,#2) ellipse (#3 and #4);}
\newcommand{\tikztellipse}[4]{\draw[thick] (#1,#2) ellipse (#3 and #4);}
\newcommand{\tikzutellipse}[4]{\draw[ultra thick] (#1,#2) ellipse (#3 and #4);}

%
% Inputs: starting point, starting angle, final angle, radius
%

\newcommand{\tikzarc}[5]{\draw (#1,#2) arc (#3:#4:#5);}
\newcommand{\tikztarc}[5]{\draw[thick] (#1,#2) arc (#3:#4:#5);}
\newcommand{\tikzutarc}[5]{\draw[ultra thick] (#1,#2) arc (#3:#4:#5);}
\newcommand{\tikzaarc}[5]{\draw[->] (#1,#2) arc (#3:#4:#5);}
\newcommand{\tikzatarc}[5]{\draw[thick,->] (#1,#2) arc (#3:#4:#5);}
\newcommand{\tikzautarc}[5]{\draw[ultra thick,->] (#1,#2) arc (#3:#4:#5);}

\newcommand{\tikznode}[3]{\node at (#1,#2) {#3};}

\begin{document}
	\begin{center}
		{\Huge \scshape \(\gamma_5\) Schemes in \(d\)-Dimensions}
	\end{center}
	
	\vspace{0.5cm}
	
	\section{Regularization and Renormalization}
	
	\subsection{Universal Features of Dimensional Regularization}
	
	As we know, because many of the Feynman integrals appearing in loop calculations are divergent, it is important to regulate these poles; in modern calculations, the most common regularization scheme is dimensional regularization, where we perform  
	\begin{equation}
	\int\frac{d^4k_i}{(2\pi)^4}\rightarrow\mu^{2\epsilon}\int\frac{d^dk_i}{(2\pi)^d}.
	\end{equation}
	In other words, we work in \(d\)-dimensions; divergences in the integrals will then be manifested as poles in \(\epsilon=(4-d)/2\). 
	
	Because we work in \(d\)-dimensions, note that contracting two metrics yields  
	\begin{equation}
	g^{\mu\nu}g_{\mu\nu}=d,
	\end{equation}
	and so 
	\begin{align*}
	\gamma^\mu\gamma_\mu & =\frac{1}{2}g^{\mu\nu}\{\gamma_\mu,\gamma_\nu\}\\
	& =d; \numberthis
	\end{align*}
	in general, \(\gamma\) matrices can be contracted according to  
	\begin{equation}
	\gamma^\mu\gamma_{\nu_1}\dots\gamma_{\nu_n}\gamma_\mu=2\sum_{i=1}^n(-1)^{i+n}\gamma_{\nu_i}\gamma_{\nu_1}\dots\gamma_{\nu_{i-1}}\gamma_{\nu_{i+1}}\dots\gamma_{\nu_n}+(-1)^{n}d\gamma_{\nu_1}\gamma_{\nu_n}.
	\end{equation}
	
	As we know, in four-dimensions, \(\gamma_5\) is defined by  
	\begin{equation}
	\gamma_5=-i\gamma_0\gamma_1\gamma_2\gamma_3;
	\end{equation}
	however, there is no clear definition for this matrix in \(d\)-dimensions. Furthermore, it is not clear that the anticommutation relation  
	\begin{equation}
	\{\gamma_\mu,\gamma_5\}=0
	\end{equation}
	will translate directly into \(d\)-dimensions while traces of Dirac matrix chains maintain their cyclic property. There are two different schemes for handling \(\gamma_5\) in dimensional regularization: in the 't Hooft-Veltman-Breitenlohner-Maison scheme, the anticommutation relation for \(\gamma_5\) is modified while the cyclic property of the trace is maintained; on the other hand, in Kreimer's scheme, the anticommutation relation is maintained while the cyclic property of the trace is abandoned. 
	
	There are no \(d\)-dimensional analogues for either \(\gamma_5\) or the Levi-Civita tensor, \(\varepsilon_{\mu\nu\rho\sigma}\); we simply accept that the definition in Equation 1.5 along with the trace  
	\begin{equation}
	\Tr[\gamma_\mu\gamma_\nu\gamma_\rho\gamma_\sigma\gamma_5]=-4i\varepsilon_{\mu\nu\rho\sigma}
	\end{equation}
	and the additional properties  
	\begin{align*}
	\gamma_5^2 & =\mathbf{1}, \numberthisa\\
	\gamma_5^\dagger & =\gamma_5 \numberthisb
	\end{align*}
	are continued to \(d\)-dimensions. However, since \(\gamma_5\) and \(\varepsilon_{\mu\nu\rho\sigma}\) are intrinsically four-dimensional objects, it is common to separate the four- and \(-2\epsilon\)-dimensional subspaces of the Lorentz indices; we write this as  
	\begin{align*}
	k^\mu & =\bar{k}^\mu+\hat{k}^\mu, \numberthisa\\
	\gamma^\mu & =\bar{\gamma}^\mu+\hat{\gamma}^\mu, \numberthisb
	\end{align*}
	where the bar denotes the four-dimensional components and the hat denotes the \(-2\epsilon\)-components. This allows us to write down the usual four-dimensional identities for the contractions of Levi-Civita tensors, e.g.  
	\begin{align*}
	\varepsilon^{\mu\nu\rho\sigma}\varepsilon_{\mu\nu\rho\sigma} & =-24, \numberthisa\\
	\varepsilon\indices{^{\mu\nu\rho}_\alpha}\varepsilon_{\mu\nu\rho\beta} & =-6\bar{g}_{\alpha\beta}, \numberthisb\\
	\varepsilon\indices{^{\mu\nu}_{\alpha\delta}}\varepsilon_{\mu\nu\beta\eta} & =-2(\bar{g}_{\alpha\beta}\bar{g}_{\delta\eta}-\bar{g}_{\alpha\eta}\bar{g}_{\beta\delta}), \numberthisc
	\end{align*}
	where \(\bar{g}_{\mu\nu}\) denotes the four-dimensional metric. 
	
	When contracting indices of \(d\)-dimensional four-vectors, we must explicitly separate out the parts as  
	\begin{equation}
	\bar{g}_{\mu\nu}k_i^\mu k_j^\nu=k_i\ndot k_j-\hat{k}_i\ndot\hat{k}_j;
	\end{equation}
	this separation can cause some difficulties in loop calculations, e.g. numerators of the form \(\hat{k}_i\ndot\hat{k}_j\) can appear. This, however, is a solved problem at both one- and two-loops. 
	
	\section{The Na\"ive Approach}
	
	In this prescription, \(\gamma_5\) satisfies the anticommutation relation  
	\begin{equation}
	\{\gamma_5,\gamma^\mu\}=0;
	\end{equation}
	in addition to this, in order to study diagrams with odd numbers of \(\gamma_5\)'s, we must include the additional criterion that  
	\begin{equation}
	\Tr(\gamma_5\gamma^\mu\gamma^\nu\gamma^\rho\gamma^\sigma)=4i\varepsilon^{\mu\nu\rho\sigma}+\order{d-4},
	\end{equation}
	where, all indices take values \(0,1,2,3\), since the Levi-Civita tensor is inherently a 4-dimensional object. 
\end{document}